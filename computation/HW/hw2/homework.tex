\documentclass[twoside,11pt]{homework}

\coursename{COMS 4236: Computational Complexity (Fall 2018)} 

\studname{Wenbo Gao}    % YOUR NAME GOES HERE
\studmail{wg2313@columbia.edu}% YOUR UNI GOES HERE
\hwNo{2}                   % THE HOMEWORK NUMBER GOES HERE
\date{\today} % DATE GOES HERE

% Uncomment the next line if you want to use \includegraphics.
\usepackage{graphicx}
%\includegraphics[height=0.3\textheight]{hw0.pdf}
\usepackage{physics}
% \usepackage{tikz}

% \usetikzlibrary{fit,positioning,arrows,automata,calc}
% \tikzset{
%   main/.style={circle, minimum size = 15mm, thick, draw =black!80, node distance = 10mm},
%   connect/.style={-latex, thick},
%   box/.style={rectangle, minimum height = 8mm,draw=black!100}
% }

% environments: theorem[*rename], proof[*rename], 

\begin{document}
\maketitle

\section*{Problem 1}
\subsection*{(a)}
\begin{prob}
  Exercises 7.7 on page 271 and Problem 7.14 on page 272 of TC (6 points):
  Show that NP is closed under union ($\cup$), concatenation ($<>$), and star
  operation ($*$).
  ( Xi: You may want to use the alternative view of NP as languages decided by
    polynomial-time verifiers.
    Can you also show that P is closed under the star operation?
  )
\end{prob}

\begin{solution}
  \ \\
  \begin{itemize}
  \item NP is closed under union ($\cup$).
    \begin{goal}
      Given $L_1 \in$ NP, $L_2 \in$ NP.
      Show $L_1 \cup L_2 \in$ NP.
    \end{goal}
    \begin{proof}
      \[
        \begin{aligned}
        L_1 \in \text{NP} \Rightarrow \exists D_1 \in \text{all NTMs}, L(D_1) = L_1\\
        L_2 \in \text{NP} \Rightarrow \exists D_2 \in \text{all NTMs}, L(D_2) = L_2
        \end{aligned}
      \]

      Construct a NTM, $D(w)$: $\forall w \in \Sigma^*$,
      \begin{enumerate}
      \item Non-deterministically guess if $w \in L_1$ , or $w \in L_2$, or $w \notin L_1
        \land w \notin L_2$.
      \item If the guess is
        \begin{itemize}
        \item $w \in L_1$, then
          simulate $D_1$ on $w$ and return the result.
        \item $w \in L_2$, then
          simulate $D_2$ on $w$ and return the result.
        \item $w \notin L_1 \land w \notin L_2$, then
          reject
        \end{itemize}
      \end{enumerate}

      $\forall w \in L_1 \cup L_2$, then $w \in L_1 \lor w \in L_2$, there is a branch
      in $D$ that accepts.

      $\forall w \notin L_1 \cup L_2$, then $w \notin L_1 \land w \notin L_2$, all branches in
      $D$ rejects:
      \begin{itemize}
      \item branch1: $D_1(w) = 0$
      \item branch2: $D_2(w) = 0$
      \item branch3: rejects
      \end{itemize}

      Therefore, $L(D) = L_1 \cup L_2$ where $D$ is a NTM, so $L_1 \cup L_2 \in$ NP.

    \end{proof}

  \item NP is closed under concatenation ($<>$).
    \begin{goal}
      Given $L_1 \in$ NP, $L_2 \in$ NP.
      Show $L_1 <> L_2 \in$ NP.
    \end{goal}
    \begin{proof}
      \[
        \begin{aligned}
        L_1 \in \text{NP} \Rightarrow \exists D_1 \in \text{all NTMs}, L(D_1) = L_1\\
        L_2 \in \text{NP} \Rightarrow \exists D_2 \in \text{all NTMs}, L(D_2) = L_2
        \end{aligned}
      \]

      Construct a NTM, $D(w)$: $\forall w \in \Sigma^*$ where $|w| = n$,
      \begin{enumerate}
      \item Non-deterministically guess a number $n_1 \in \{0, 1, \dots, n\}$
      \item Let $w_1$ be the first $n_1$ characters of $w$, and $w_2$ be the
        rest $(n-n_1)$ characters.
        Simulate $D_1$ on $w_1$ and $D_2$ on $w_2$, and return $D_1(w_1)
        \land D_2(w_2)$.
      \end{enumerate}
      
      $\forall w \in L_1 <> L_2$, there is a branch in $D$ that accepts where
      the choice of $n_1$ is exactly the length of $w_1 \in L_1$ and $(n - n_1)$ is
      exactly the length of $w_2 \in L_2$ and $w = w_1 <> w_2$.

      $\forall w \notin L_1 <> L_2$, all branches in
      $D$ rejects since there's no such a partition of $w$ into $w_1$ and $w_2$
      where $D_1(w_1)=1 $ and $D_2(w_2)=1$.

      Therefore, $L(D) = L_1 <> L_2$ where $D$ is a NTM, so $L_1 <> L_2 \in$ NP.
    \end{proof}

  \item NP is closed under star operation ($*$).

    \begin{goal}
      Given $L_1 \in$ NP.
      Show $L_1^* \in$ NP.
    \end{goal}
    \begin{proof}
      \[
        \begin{aligned}
        L_1 \in \text{NP} \Rightarrow \exists D_1 \in \text{all NTMs}, L(D_1) = L_1\\
        \end{aligned}
      \]

      Construct a NTM, $D(w)$: $\forall w \in \Sigma^*$ where $|w| = n$,\\
      Non-deterministically guess a number $k \in \{0, 1, \dots, n\}$
      \begin{itemize}
      \item if $n \not\equiv 0 (\text{mod}\ k)$, then reject
      \item if $n \equiv 0 (\text{mod}\ k)$, then let $x$ be the first $k$ characters
        of $w$.\\
        Simulate $D_1$ on $x$ and return $D_1(x)$.
      \end{itemize}

      $\forall w \in L_1^*$, there is a branch in $D$ that accepts.

      $\forall w \notin L_1^*$, all branches in $D$ rejects.

      Therefore, $L(D) = L_1^*$ where $D$ is a NTM, so $L_1^* \in$ NP.
    \end{proof}
  \end{itemize}
\end{solution}

\subsection*{(b)}

\begin{prob}
  Exercise 7.11 on page 272 of TC (4 points):
  Call graph $G$ and $H$ isomorphic if the nodes of G may be reordered so that
  it is identical to $H$.
  Let
  \[
    \text{ISO} = \{ (G, H) \ | \ G \text{ and } H \text{ are isomorphic graphs} \}.
  \]
  Show that ISO $\in$ NP.
\end{prob}
\begin{proof}
  Construct a NTM, $D(w)$:
  $\forall G(V_G, E_G), H(V_H, E_H) \in$ all graphs,
  \begin{itemize}
  \item If $|V_G| \ne |V_H| \lor |E_G| \ne |E_H|$, reject
  \item Otherwise, let $|V_G| = |V_H| = n$.
    \begin{enumerate}
    \item Enumerate all vertices in $V_G$, and non-deterministically guess a
    new assignment of index $\in \{1,2,\dots,n\}$ to each vertex.
    (Here I assume vertices in G and H are indexed by numbers from $\{1,2,\dots,
    n\}$)
    \item Once we have the reordered graph $V_G'$, compare $V_G'$ and $H$ by
      vertices and edges.\\
      If $V_G'$ and $H$ are identical, accept; otherwise, reject.
    \end{enumerate}
  \end{itemize}

  $\forall (G,H) \in ISO$, there is a branch in $D$ that accepts.

  $\forall (G,H) \notin ISO$, all branches in $D$ rejects.

  Therefore, $L(D) = $ ISO where $D$ is a NTM, so ISO $\in$ NP.
\end{proof}

\section*{Problem 2}

\subsection*{(a)}

\begin{prob}
  Problem 8.20 on page 304 of TC (10 points):
  An undirected graph is bipartite if its nodes may be divided into two sets so
  that all edges go from a node in one set to a node in the other set.
  Show that a graph is bipartite if and only if it doesn't contain a cycle that
  has an odd number of nodes.
\end{prob}

\begin{goal}
  Prove $G(V,E)$ is a bipartite graph $\Leftrightarrow$ $\forall p \subseteq E$, if $p$ is a cycle, then $|p|$ is even.
\end{goal}

\begin{proof}
  \ \\
  \begin{itemize}
  \item LHS $\Rightarrow$ RHS
    \ \\
    A cycle is a closed path which starts and ends at the same vertex.
    Thus, in a bipartite graph, a cycle starts and ends at the same side.
    In order to get back to the same side, it takes even-number steps,
    i.e. Left$\rightarrow$Right, and then Right$\rightarrow$Left, because there's no edge within the
    same side. So all cycles in a bipartite graph are of even length. Since
    cycles have the same number of edges as the number of vertices, all cycles
    in a bipartite graph have even number of vertices.
  \item LHS $\Leftarrow$ RHS
    Pick a vertex $v_0 \in V$.
    Partition $V$ into two sets:
    \begin{enumerate}
    \item vertices of even-length distance from $v_0$, including $v_0$ itself
    \item vertices of odd-length distance from $v_0$
    \end{enumerate}

    Proof by contradiction:\\
    Assume there exists edges in the same set, then there exists an odd cycle
    containing $v_0$ and there is a vertex in the same set as $v_0$ but of
    odd-length distance to $v_0$, which leads to a contradiction.
    Therefore, there's no edges in the same set and thus this graph is a bipartite graph.
  \end{itemize}
  
\end{proof}

\subsection*{(b)}
\begin{prob}
  Let
  \[
    \text{BIPARTITE} = \{ G \ |\ G \text{ is a bipartite graph} \}.
  \]
  Show that BIPARTITE $\in$ NL.
  ( Xi: What is the more natural complexity class to which BIPARTITE belongs?
    as suggested by the fact mentioned, and what do we know about this class?
  )
\end{prob}

\begin{proof}
  BIPARTITE $\in$ NL.\\

  Since NL = CoNL, it's equivalent to prove $\overline{\text{BIPARTITE}} \in$ NL.

  Construct a log-space NTM, $D(G)$: $\forall G(V,E) \in \text{all graphs}$,
  \begin{enumerate}
  \setcounter{enumi}{-1}
  \item Initialize a counter: $i = 1$
  \item Non-deterministically decide a vertex, $v_1 \in V$, as the starting point.\\
    Increment counter, $i := i + 1 \Rightarrow i = 2$.
  \item Non-deterministically decide a vertex, $v_i \in V$,
    \begin{itemize}
    \item if $(V_i,V_{i-1}) \notin E$, reject.
    \item if $(V_i,V_{i-1}) \in E$,
      \begin{itemize}
      \item if $V_i = V_1$ and $i$ is odd, then accept;
      \item if $i > |V|$, then reject;
      \item if $V_i \ne V_1$, increment counter, $i := i + 1$, and repeat step 2.
      \end{itemize}
    \end{itemize}
  \end{enumerate}

  $\forall G \in \overline{\text{BIPARTITE}}$, there exists an odd cycle in $G$, then
  there is one branch in $D$ that accepts.

  $\forall G \notin \overline{\text{BIPARTITE}}$, then no branch in $D$ accepts.

  Therefore, $L(D) = \overline{\text{BIPARTITE}}$ where $D$ is a log-space NTM,
  so $\overline{\text{BIPARTITE}} \in$ NL and thus BIPARTITE $\in$ NL.

\end{proof}

\section*{Problem 3}

\begin{prob}
  Show that P $\ne$ SPACE(n).
  ( Hint: Assume P = SPACE(n).
    Use the space hierarchy theorem to derive a contradition.
    You may find the function pad, defined in Problem 9.18, to be helpful.
  )
\end{prob}
\begin{proof}[Proof by contradiction]
  Assume P = SPACE(n), then $\forall L \in$ SPACE(n), $\exists D \in$ all DTMs with time
  complexity $p(n)$, $L(D) = L$, where $p$ is a polynomial function.

  Then for SPACE($n^2$), $\forall L \in$ SPACE(n), $\exists D' \in$ all DTMs with time
  complexity $p(n^2)$ which is still polynomial, $L(D') = L$.
  Therefore, P = SPACE($n^2$) and thus SPACE(n) = SPACE($n^2$) which contradicts
  the result, SPACE(n) $\ne$ SPACE($n^2$), from space hierarchy theorem.

  So P $\ne$ SPACE(n).

\end{proof}

\section*{Problem 4}

\begin{prob}
  Problem 8.15 on page 303 of TC (5 points if you can show the following problem
  is in PSPACE; 10 points if you can show if it is in P!):
  The cat-and-mouse game is played by two players, "CAT" and "Mouse", on an
  arbitrary undirected graph.
  At a given point each player occupies a node of the graph.
  The players take turns moving to a node adjacent to the one that they
  currently occupy.
  A special node of the graph is called "Hole".
  Terminal conditions of the game:
  \begin{itemize}
  \item \textbf{Cat wins} if the two players ever occupy the same node.
  \item \textbf{Mouse wins} if it reaches the Hole before the preceding happens.
  \item The game is a \textbf{draw} if the two players ever simultaneously reach
    positions that they previously occupied.
  \end{itemize} 
  Let
  \[
    \begin{aligned}
      \text{HAPPY-CAT} = \{ (G,c,m,h) \ | \
      &\text{G,c,m,h are respectively a graph, and}\\
      &\text{positions of the Cat, Mouse, and Hole, such that}\\
      &\text{Cat has a wining strategy, if Cat moves first}.\}
    \end{aligned}
  \]
  Show that HAPPY-CAT is in PSPACE (and in P for more points).
\end{prob}

\begin{proof}
  HAPPY-CAT is in P. \\
  Given undirected graph $G(V,E)$, where $|V| = n$.\\
  The state space of the game has $2n^2$ vertices:
  \[
    \begin{aligned}
    \text{State} =&\\
      &\{ &\text{Cat\_position} &:: V\\
      &,  &\text{Mouse\_position} &:: V\\
      &,  &\text{turn\_player} &:: \{ \text{Cat}, \text{Mouse}\}\\
      &\}
    \end{aligned}
  \]
  \[
    \# \text{ of states} = |V| * |V| * 2 = n * n * 2 = 2n^2
  \]
  The edges between states are valid state transitions, the number of which is
  at most $8n^4$.

  Each vertex takes $log_2n$ bits and each edge takes $2log_2n$ bits, so the
  entire state graph takes $(2n^2log_2n + 16n^4log_2n)$ bits which is $O(n^5)$.

  Annotate the states by Cat's win condition:
  \[
    \forall V_0 \in V, (\{V_0, V_0, \text{Cat}\}, \text{Cat\_Win})
    \text{ and } 
    (\{V_0, V_0, \text{Mouse}\}, \text{Cat\_Win})
  \]

  HAPPY-CAT then can be formalized as a reachability problem:\\
  Given Cat's initial position $V_{Cat\_init} \in V$,
  Mouse's initial position $V_{Mouse\_init} \in V$,
  and initial turn player Cat, whether there's a state annotated by Cat\_Win can
  be reached from this initial state.

  We can use BFS traversing the state graph backwards from Cat's win states,
  until reach the initial state and accept, or exhaust all the reachable edges
  without hitting the initial state and reject.
  The runtime is bounded by the number of edges which is $O(n^4)$.

  Therefore, HAPPY-CAT is in P.
\end{proof}


\end{document} 
