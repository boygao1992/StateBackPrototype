\documentclass[twoside,11pt]{homework}

\coursename{COMS 4236: Computational Complexity (Fall 2018)} 

\studname{Wenbo Gao}    % YOUR NAME GOES HERE
\studmail{wg2313@columbia.edu}% YOUR UNI GOES HERE
\hwNo{2}                   % THE HOMEWORK NUMBER GOES HERE
\date{\today} % DATE GOES HERE

% Uncomment the next line if you want to use \includegraphics.
\usepackage{graphicx}
%\includegraphics[height=0.3\textheight]{hw0.pdf}
\usepackage{physics}
% \usepackage{tikz}

% \usetikzlibrary{fit,positioning,arrows,automata,calc}
% \tikzset{
%   main/.style={circle, minimum size = 15mm, thick, draw =black!80, node distance = 10mm},
%   connect/.style={-latex, thick},
%   box/.style={rectangle, minimum height = 8mm,draw=black!100}
% }

% environments: theorem[*rename], proof[*rename], 

\begin{document}
\maketitle

\section*{Problem 4}

\begin{prob}
  Problem 8.15 on page 303 of TC (5 points if you can show the following problem
  is in PSPACE; 10 points if you can show if it is in P!):
  The cat-and-mouse game is played by two players, "CAT" and "Mouse", on an
  arbitrary undirected graph.
  At a given point each player occupies a node of the graph.
  The players take turns moving to a node adjacent to the one that they
  currently occupy.
  A special node of the graph is called "Hole".
  Terminal conditions of the game:
  \begin{itemize}
  \item \textbf{Cat wins} if the two players ever occupy the same node.
  \item \textbf{Mouse wins} if it reaches the Hole before the preceding happens.
  \item The game is a \textbf{draw} if the two players ever simultaneously reach
    positions that they previously occupied.
  \end{itemize} 
  Let
  \[
    \begin{aligned}
      \text{HAPPY-CAT} = \{ (G,c,m,h) \ | \
      &\text{G,c,m,h are respectively a graph, and}\\
      &\text{positions of the Cat, Mouse, and Hole, such that}\\
      &\text{Cat has a wining strategy, if Cat moves first}.\}
    \end{aligned}
  \]
  Show that HAPPY-CAT is in PSPACE (and in P for more points).
\end{prob}

\begin{solution}
\begin{proof}[HAPPY-CAT is in P]
  \ \\
  Given undirected graph $G(V,E)$, where $|V| = n$.\\
  The state space of the game has $2n^2$ vertices:
  \[
    \begin{aligned}
    \text{State} =&\\
      &\{ &\text{Cat\_position} &:: V\\
      &,  &\text{Mouse\_position} &:: V\\
      &,  &\text{turn\_player} &:: \{ \text{Cat}, \text{Mouse}\}\\
      &\}
    \end{aligned}
  \]
  \[
    \# \text{ of states} = |V| * |V| * 2 = n * n * 2 = 2n^2
  \]
  The edges between states are valid state transitions, the number of which is
  at most $8n^4$.

  Each vertex takes $log_2n$ bits and each edge takes $2log_2n$ bits, so the
  entire state graph takes $(2n^2log_2n + 16n^4log_2n)$ bits which is $O(n^5)$.

  Annotate the states by Cat's win condition:
  \[
    \forall V_0 \in V, (\{V_0, V_0, \text{Cat}\}, \text{Cat\_Win})
    \text{ and } 
    (\{V_0, V_0, \text{Mouse}\}, \text{Cat\_Win})
  \]

  HAPPY-CAT then can be formalized as a reachability problem:\\
  Given Cat's initial position $V_{Cat\_init} \in V$,
  Mouse's initial position $V_{Mouse\_init} \in V$,
  and initial turn player Cat, whether there's a state annotated by Cat\_Win can
  be reached from this initial state.

  We can use BFS traversing the state graph backwards from Cat's win states,
  until reach the initial state and accept, or exhaust all the reachable edges
  without hitting the initial state and reject.
  The runtime is bounded by the number of edges which is $O(n^4)$.

  Therefore, HAPPY-CAT is in P.
\end{proof}
\end{solution}


\end{document} 
