\documentclass[twoside,11pt]{homework}

\coursename{COMS 4236: Computational Complexity (Fall 2018)} 

\studname{Wenbo Gao}    % YOUR NAME GOES HERE
\studmail{wg2313@columbia.edu}% YOUR UNI GOES HERE
\hwNo{3}                   % THE HOMEWORK NUMBER GOES HERE
\date{\today} % DATE GOES HERE

% Uncomment the next line if you want to use \includegraphics.
\usepackage{graphicx}
%\includegraphics[height=0.3\textheight]{hw0.pdf}
\usepackage{physics}
% \usepackage{tikz}

% \usetikzlibrary{fit,positioning,arrows,automata,calc}
% \tikzset{
%   main/.style={circle, minimum size = 15mm, thick, draw =black!80, node distance = 10mm},
%   connect/.style={-latex, thick},
%   box/.style={rectangle, minimum height = 8mm,draw=black!100}
% }

% environments: theorem[*rename], proof[*rename], 

\begin{document}
\maketitle

\section*{Problem 1}

\begin{prob}
  Problem 11.5.18 on page 275 of CC:
  Show that, if NP $\subseteq$ BPP, then RP = NP.
  (That is, if SAT can be solved by randomized machine, then it can be solved by
  randomized machines with no false positives, presumably by computing a
  satisfying truth as in Example 10.3.)
\end{prob}

\section*{Problem 2}
\begin{prob}
  Let $0 < \epsilon_1 < \epsilon_2 < 1$ denote two constants. Let $D(\cdot, \cdot)$ be a deterministic
  polynomial-time computable Boolean function, and let $L$ be a language (the
  setting so far is exactly the same as the definition of BPP.)
  $D$ and $L$ satisfy the following property:
  Given any $x \in \{0, 1\}^n$, if $y$ is sampled uniformly at random from $\{ 0,1
  \}^m$ for some $m$ polynomial in $n$, then
  \[
    x \in L \Rightarrow \text{Pr}_{y \in \{0,1\}^m}[D(x,y) = 1] \ge \epsilon_2
    \text{ and }
    x \notin L \Rightarrow \text{Pr}_{y \in \{0,1\}^m}[D(x,y) = 1] \le \epsilon_1.
  \]

  Show that $L \in$ BPP.
  (Note that $\epsilon_2$ can be smaller than $1/2$.
   Use the Chernoff bound.)
\end{prob}

\section*{Problem 3}

\begin{prob}
  Similar to P/poly, one can define P/$\log n$, where the advice string is of
  length only $O(\log n)$ for input size $n$.
  Show that, if SAT $\in$ P/$\log n$, then P = NP.
  (Hint: Self-reducibility.)
\end{prob}

\section*{Problem 4}

\begin{prob}
  Show that, if PSPACE $\subseteq$ P/poly, then PSPACE = $\Sigma_2^P$.
  (Hint: Use self-reducibility to "implicitly" build a winning strategy for the
  existential player in the TQBF game.)
\end{prob}

\end{document} 
