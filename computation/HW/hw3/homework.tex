\documentclass[twoside,11pt]{homework}

\coursename{COMS 4236: Computational Complexity (Fall 2018)} 

\studname{Wenbo Gao}    % YOUR NAME GOES HERE
\studmail{wg2313@columbia.edu}% YOUR UNI GOES HERE
\hwNo{3}                   % THE HOMEWORK NUMBER GOES HERE
\date{\today} % DATE GOES HERE

% Uncomment the next line if you want to use \includegraphics.
\usepackage{graphicx}
%\includegraphics[height=0.3\textheight]{hw0.pdf}
\usepackage{physics}
% \usepackage{tikz}

% \usetikzlibrary{fit,positioning,arrows,automata,calc}
% \tikzset{
%   main/.style={circle, minimum size = 15mm, thick, draw =black!80, node distance = 10mm},
%   connect/.style={-latex, thick},
%   box/.style={rectangle, minimum height = 8mm,draw=black!100}
% }

% environments: theorem[*rename], proof[*rename], 

\begin{document}
\maketitle

\section*{Problem 1}

Problem 11.5.18 on page 275 of CC:
Show that, if NP $\subseteq$ BPP, then RP = NP.
(That is, if SAT can be solved by randomized machine, then it can be solved by
randomized machines with no false positives, presumably by computing a
satisfying truth as in Example 10.3.)

\begin{goal}
  Prove SAT $\in$ NP-complete $\subseteq$ NP $\subseteq$ RP.
\end{goal}

\begin{proof}
  \ \\
  Let $A$ be a machine in BPP that decides SAT, $L(A) = SAT$,
  \begin{given}
    \[
      \begin{aligned}
        \forall x \in SAT \Rightarrow \text{Pr}_{y \in \{0,1\}^{q(n)}}[D(x,y) = 1] \ge \frac{2}{3}\\
        \forall x \notin SAT \Rightarrow \text{Pr}_{y \in \{0,1\}^{q(n)}}[D(x,y) = 1] \le \frac{1}{3}
      \end{aligned}
    \]
  \end{given}
  $\forall w \in SAT$, where w is binary encoding of a boolean formula $\Phi$,\\
  where $w = < \lambda x_1, x_2, \dots, x_n. \Phi(x_1, x_2, \dots, x_n)>$,\\
  construct a DTM, $D$, as the following,
  \begin{enumerate}
  \item Initialize counter $i = 1$
  \item Simulate both $\lambda x_{i+1}, \dots, x_n. \Phi(x_1, \dots, True,x_{i+1}, \dots, x_n)$
    and\\ $\lambda x_{i+1}, \dots, x_n. \Phi(x_1, \dots, False, x_{i+1}, \dots, x_n)$ on $A$
  \item If $(A(\Phi(x_1, \dots, T,x_{i+1}, \dots, x_n)), A(\Phi(x_1, \dots, F, x_{i+1}, \dots, x_n))$
    \begin{itemize}
    \item $= (0,0)$, then reject
    \item $= (1,0)$, then keep $x_{i} = True$, increment counter $i = i + 1$, and repeat step 2
    \item $= (0,1)$, then keep $x_{i} = False$ , increment counter $i = i + 1$, and repeat step 2
    \item $= (1,1)$, then keep $x_{i} = True$ (or $x_{i} = False$, the choice doesn't
      matter), increment counter $i = i + 1$, and repeat step 2.
    \end{itemize}
  \item Once we have an assignment of $\Phi$, verify the assignment
    \begin{itemize}
    \item If $\Phi(x_1, x_2, \dots, x_n) = 1$, then accept;
    \item otherwise, reject.
    \end{itemize}
  \end{enumerate}

  This way we guarantee that if $w \notin SAT$, our machine $D$ always rejects.
  
  If $w \in SAT$, there is $\ge (\frac{2}{3})^n$ chance that the machine $A$ is
  correct in all $n$ iterations.

  We can amplify this probability to be $\ge \frac{1}{2}$ by repeating this
  experiment polynomial times.
  
  Therefore, $L(D) =$ SAT $\land L(D) \in$ RP, and thus SAT $\in$ RP.
\end{proof}

\section*{Problem 2}
Let $0 < \epsilon_1 < \epsilon_2 < 1$ denote two constants. Let $D(\cdot, \cdot)$ be a deterministic
polynomial-time computable Boolean function, and let $L$ be a language (the
setting so far is exactly the same as the definition of BPP.)
$D$ and $L$ satisfy the following property:
Given any $x \in \{0, 1\}^n$, if $y$ is sampled uniformly at random from $\{ 0,1
\}^m$ for some $m$ polynomial in $n$, then
\[
  x \in L \Rightarrow \text{Pr}_{y \in \{0,1\}^m}[D(x,y) = 1] \ge \epsilon_2
  \text{ and }
  x \notin L \Rightarrow \text{Pr}_{y \in \{0,1\}^m}[D(x,y) = 1] \le \epsilon_1.
\]

Show that $L \in$ BPP.
(Note that $\epsilon_2$ can be smaller than $1/2$.
  Use the Chernoff bound.)

\begin{proof}
  Construct a DTM, $D'$, which independently draws $k$ instances from
  $\{0,1\}^m$,\\ $\{y_1, y_2, \dots, y_k\}$, $\forall x \notin L$,
  \[
    E[ \sum_{i = 1}^k D(x,y_i) ] = k \cdot \epsilon_1
  \]
  By Chernoff bound, $\Delta = k \cdot (\frac{\epsilon_1 + \epsilon_2}{2} - \epsilon_1) = k \cdot \frac{\epsilon_2 - \epsilon_1}{2}$,
  \[
    \begin{aligned}
      Pr[D'(x, y_1, y_2, \dots, y_k) = 1]
      = Pr[ \sum_{i = 1}^k D(x,y_i) \ge k \cdot \frac{\epsilon_1 + \epsilon_2}{2}]
      \le e^{\frac{-2 (k \cdot \frac{\epsilon_2 - \epsilon_1}{2})^2}{k}} = e^{-\Omega(k)}
    \end{aligned}
  \]

  Therefore, $L = L(D') \in$ Strong BPP, and thus $L \in$ BPP.
\end{proof}

\section*{Problem 3}

\begin{prob}
  Similar to P/poly, one can define P/$\log n$, where the advice string is of
  length only $O(\log n)$ for input size $n$.
  Show that, if SAT $\in$ P/$\log n$, then P = NP.
  (Hint: Self-reducibility.)
\end{prob}

\begin{proof}
  Since advice string is of $O(\log n)$, it's able to enumerate all 
  $2^{O(\log n)} = O(n)$ advice strings in poly-time and, $\forall x \in \{0,1\}^n$,
  if at least one advice string accepts then accept; otherwise, reject.
  
  This way we can determinitically decide SAT in poly-time.
  Therefore, P = NP.
\end{proof}

\section*{Problem 4}

Show that, if PSPACE $\subseteq$ P/poly, then PSPACE = $\Sigma_2^P$.
(Hint: Use self-reducibility to "implicitly" build a winning strategy for the
existential player in the TQBF game.)

\begin{proof}
  PSPACE $\subseteq$ P/poly $\Rightarrow$ there exists a poly-size circuit sequence $\{C_m\}$ where
  $C_m$ decides TQBF instances of size $m$.

  For a TQBF instance of size $m$, $Q_1 X_1. Q_2 X_2. \dots Q_n X_n. \Phi(X_1,
  \dots, X_n)$,\\
  use $\exists$ quantifier of $\Sigma_2^p$ to non-deterministically guess the circuit
  sequence,
  \begin{enumerate}
  \item initialize counter $i = 1$
  \item for $\exists$-player's turn, $\exists X_i$, check $C_k(Q_{i+1}X_{i+1}. \dots Q_n X_n.
    \Phi(X_1, \dots, True,
    X_{i+1}, \dots, X_n))$\\
    and $C_k(Q_{i+1}X_{i+1}. \dots Q_n X_n. \Phi(X_1, \dots, False, X_{i+1}, \dots, X_n))$,
    where $k \le m$,
    \begin{itemize}
    \item if both $= 0$, then reject
    \item Otherwise, keep the assignment of $X_i$ which leads to $C_k(\dots) =
      1$, increment counter $i = i + 1$, and repeat step 2 - 3
    \end{itemize}
  \item for $\forall$-player's turn, use $\forall$ quantifier of $\Sigma_2^p$
  \end{enumerate}

  Therefore, TQBF $\in \Sigma_2^P$ and thus PSPACE = $\Sigma_2^P$.
\end{proof}

\end{document} 
